\documentclass[a4paper, 11pt]{article}
\usepackage{comment} % enables the use of multi-line comments (\ifx \fi) 
\usepackage{lipsum} %This package just generates Lorem Ipsum filler text. 
\usepackage{fullpage} % changes the margin
\usepackage{amsmath}

\begin{document}

\noindent
\large\textbf{Tasks 2\&3} \hfill \textbf{17-941-998}

\section*{Auxiliary Variables}
To assist interpretation, the following auxiliary variables are added to the buffer:
\begin{itemize}
    \item \texttt{t\_used, t\_valid}: the tail, split into $t_\text{used}$ (the item exactly before it is allocated to a writer, but not yet ready to pop) and $t_\text{valid}$ (the items before it are ready for a reader to pop).
    \item \texttt{h\_used, h\_red, h\_blue}: similarly, the head is also split, $h_\text{used}$ means allocated to a reader, while $h_\text{valid}$ is further split into $h_\text{red}$ and $h_\text{blue}$, indicating the number of items that are finished reading for each stream, and $h_\text{red} + h_\text{blue} = h_\text{valid}$. $h_\text{valid}$ is thus implicit.
    \item \texttt{x\_red, x\_blue}: the x pointers in both input streams are copied into the buffer so that x can be shared by both streams to help interpretation. The original x pointers (private in each stream) are kept but not used in the interpretations.
\end{itemize}

\noindent Added to the streams are:
\begin{itemize}
    \item \texttt{Red.H, Red.T}: booleans, meaning ``Red owns the Head pointer/Tail pointer'';
    \item \texttt{Blue.H, Blue.T}: likewise.
\end{itemize}

\section*{New Notations}
To distinguish items of different colors for buffer interpretation, the following syntax is defined using our former \texttt{upto} notation and Python-like list comprehension:
\begin{itemize}
    \item \texttt{B[h redupto t) = [item.v for item in B[h upto t) if item.c = Red]}
    \item \texttt{B[h blueupto t) = [item.v for item in B[h upto t) if item.c = Blue]}
\end{itemize}

\section*{Interpretations}
\begin{flalign*}
Red.S_O &= Red.O[0, Red.y)
&Blue.S_O &= Blue.O[0, Blue.y) \\
Red.S_{BO} &= Red.O[Red.y, h_\text{red})
&Blue.S_{BO} &= Blue.O[Blue.y, h_\text{blue}) \\
Red.S_B &= B[(h_\text{red} + h_\text{blue})\ \text{redupto}\ t_\text{valid}) 
&Blue.S_B &= B[(h_\text{red} + h_\text{blue})\ \text{blueupto}\ t_\text{valid})  \\
Red.S_{IB} &= B[t_\text{valid}\ \text{redupto}\ (x_\text{red}+x_\text{blue})) 
&Blue.S_{IB} &= B[t_\text{valid}\ \text{blueupto}\ (x_\text{red}+x_\text{blue}))   \\
Red.S_I &= Red.I[x_\text{red}, Red.N)
& Blue.S_I &= Blue.I[x_\text{blue}, Blue.N)
\end{flalign*}

\section*{Invariant}

The invariant can be split into five parts:

\begin{flalign*}S:\ &
\text{Red}.S_O+
\text{Red}.S_{BO}+
\text{Red}.S_B+
\text{Red}.S_{IB}+
\text{Red}.S_I = K_\text{Red}
\land \\
& \text{Blue}.S_O+
\text{Blue}.S_{BO}+
\text{Blue}.S_B+
\text{Blue}.S_{IB}+
\text{Blue}.S_I = K_\text{Blue} \\
B:\ &
h_\text{Red} + h_\text{Blue} \leq t_\text{valid} \land t_\text{valid} - h_\text{Red} - h_\text{Blue} \leq N_B \\
I:\ &
\text{Red}.x \leq \text{Red}.N \land \text{Blue}.x \leq \text{Blue}.N \\
O:\ &
\text{Red}.y \leq \text{Red}.N \land \text{Blue}.y \leq \text{Blue}.N \\
T:\ &
\neg (\text{Red}.H \land \text{Blue}.H) \land \neg (\text{Red}.T \land \text{Blue}.T)
\end{flalign*}

S ensures for both streams the overall content is constant. B is the boundary invariant for the buffer. I and O ensures the boundary of input and output pointers for both streams. T is the mutual exclusion invariant so that each pointer can be held by at most one stream.

\subsection*{Local Correctness}

For brevity, the annotations are only shown for \texttt{Red.push()} and \texttt{Red.pop()}. For Blue it is only reversing the color.

All variables are assigned 0 (or \texttt{false} for booleans) on initialization.

For \texttt{push()}, $\text{Red}.T \rightarrow t_\text{valid} = x_\text{red} + x_\text{blue}$ is true on initialization, and it is proved as below that it is preserved by \texttt{push()}, thus by induction it is a precondition and a postcondition. The same reasoning applies for the precondition and postcondition \texttt{Red.y = h\_red} in \texttt{pop()}.

\begin{verbatim}
Red.push()
[Precondition] Red.T->(t_valid=x_red+x_blue), Red.x<Red.N, not Red.T
[Postcondition] Red.T->(t_valid=x_red+x_blue), Red.x<=Red.N, not Red.T

> S, B, I, O, T holds
> Red.x < Red.N and not Red.T
> Red.T -> (t_valid = x_red+x_blue)
do (not Red.T)
  > S, B, I, O, T, Red.x < Red.N, not Red.T, Red.T->(t_valid=x_red+x_blue)
  t' := t_used
  > S, B, I, O, T, Red.x < Red.N, not Red.T
  > Red.T -> (t_valid = x_red + x_blue)
  if t' = t_valid then
    > S, B, I, O, T, Red.x < Red.N, not Red.T, Red.T->(t_valid=x_red+x_blue)
    > t' = t_used -> t_valid = t_used
    if t’ - h_red - h_blue < N_B then
      > S, B, I, O, T, Red.x < Red.N, not Red.T, Red.T->(t_valid=x_red+x_blue)
      > t' = t_used -> (t_valid = t_used and t_valid - h_red - h_blue < N_B)
      CAS(Red.T, t_used, t'+1, t')
      > S, B, I, O, T, Red.x < Red.N
      > Red.T -> (t_valid = x_red + x_blue)
      > Red.T -> (t_valid + 1 = t_used and t_valid - h_red - h_blue < N_B)
      > Red.T -> not Blue.T
    fi
  fi
  > S, B, I, O, T, Red.x < Red.N
  > Red.T -> (t_valid = x_red + x_blue)
  > Red.T -> (t_valid + 1 = t_used and t_valid - h_red - h_blue < N_B)
  > Red.T -> not Blue.T
od
> S, B, I, O, T, Red.x < Red.N, Red.T, not Blue.T
> t_valid = x_red + x_blue
> t_valid + 1 = t_used, t_valid - h_red - h_blue < N_B
> Red.S_O + Red.S_BO + B[(h_red+h_blue) redupto t_valid) + [] +
  (Red.I[x_red] + Red.I[x_red + 1, Red.N)) = K_Red
B[t_valid mod N_B].v := Red.I[Red.x]
> S, B, I, O, T, Red.x < Red.N, Red.T, not Blue.T
> t_valid = x_red + x_blue
> t_valid + 1 = t_used, t_valid - h_red - h_blue < N_B
> Red.S_O + Red.S_BO + B[(h_red+h_blue) redupto t_valid) + [] +
  (B[t_valid mod N_B].v + Red.I[x_red + 1, Red.N)) = K_Red
B[t_valid mod N_B].c := Red
> S, B, I, O, T, Red.x < Red.N, Red.T, not Blue.T
> t_valid = x_red + x_blue
> t_valid + 1 = t_used, t_valid - h_red - h_blue < N_B
> Red.S_O + Red.S_BO + B[(h_red+h_blue) redupto t_valid) + [] +
  (B[t_valid mod N_B].v + Red.I[x_red + 1, Red.N)) = K_Red
x_red := x_red + 1
> S, B, I, O, T, Red.x < Red.N, Red.T, not Blue.T
> t_valid + 1 = x_red + x_blue
> t_valid + 1 = t_used, t_valid - h_red - h_blue < N_B
> Red.S_O + Red.S_BO + B[(h_red+h_blue) redupto t_valid) +
  B[t_valid mod N_B].v + Red.I[x_red, Red.N) = K_Red
Red.x := Red.x + 1
> S, B, I, O, T, Red.x <= Red.N, Red.T, not Blue.T
> t_valid + 1 = x_red + x_blue
> B[(t_valid + 1) redupto (x_red + x_blue)] = []
> t_valid + 1 = t_used, t_valid - h_red - h_blue < N_B
> Red.S_O + Red.S_BO + B[(h_red+h_blue) redupto t_valid) +
  B[t_valid mod N_B].v + Red.I[x_red, Red.N) = K_Red
Red.T := false
> S, B, I, O, T, Red.x <= Red.N, not Red.T, not Blue.T
> B[(t_valid + 1) redupto (x_red + x_blue)] = []
> t_valid + 1 = t_used, t_valid - h_red - h_blue < N_B
> Red.S_O + Red.S_BO + B[(h_red+h_blue) redupto t_valid) +
  B[t_valid mod N_B].v + Red.I[x_red, Red.N) = K_Red
t_valid := t_valid + 1
> S, B, I, O, T, Red.x <= Red.N, not Red.T
> Red.T -> (t_valid = x_red + x_blue)
> B[t_valid redupto (x_red + x_blue)] = []
> Red.S_O + Red.S_BO + B[(h_red+h_blue) redupto t_valid) + [] +
  Red.I[x_red, Red.N) = K_Red
\end{verbatim}


\begin{verbatim}
Red.pop()
[Precondition] Red.y = h_red and Red.y < Red.N and not Red.H
[Postcondition] Red.y = h_red and Red.y <= Red.N and not Red.H

> S, B, I, O, T holds
> Red.y = h_red, Red.y < Red.N, not Red.H
do (not Red.H)
  > S, B, I, O, T, Red.y = h_red, Red.y < Red.N, not Red.H
  h' := h_used
  > S, B, I, O, T, Red.y = h_red, Red.y < Red.N, not Red.H
  if h' = h_red + h_blue then
    > S, B, I, O, T, Red.y = h_red, Red.y < Red.N, not Red.H
    > h' = h_used -> h_used = h_red + h_blue
    if h' < t_valid and B[h' mod N_B].c = Red then
      > S, B, I, O, T, Red.y = h_red, Red.y < Red.N, not Red.H
      > h' = h_used -> h_used = h_red + h_blue
      > h' = h_used -> h_used < t_valid and B[h_used mod N_B].c = Red
      CAS(Red.H, h_used, h'+1, h')
      > S, B, I, O, T, Red.y = h_red, Red.y < Red.N
      > Red.H -> h_used = h_red + h_blue + 1
      > Red.H -> h_red + h_blue < t_valid
      > Red.H -> B[(h_red + h_blue) mod N_B].c = Red
      > Red.H -> not Blue.H
    fi
  fi
  > S, B, I, O, T, Red.y = h_red, Red.y < Red.N
  > Red.H -> h_used = h_red + h_blue + 1
  > Red.H -> h_red + h_blue < t_valid
  > Red.H -> B[(h_red + h_blue) mod N_B].c = Red
  > Red.H -> not Blue.H
od
> S, B, I, O, T, Red.y = h_red, Red.y < Red.N, Red.H, not Blue.H
> h_used = h_red + h_blue + 1
> h_red + h_blue < t_valid
> B[(h_red + h_blue) mod N_B].c = Red
> Red.O[0, Red.y) + [] + (B[(h_red + h_blue) mod N_B].v + 
  B[(h_red + h_blue + 1) redupto t_valid)) + Red.S_IB + Red.S_I = K_Red
Red.O[Red.y] := B[(h_red + h_blue) mod N_B].v
> S, B, I, O, T, Red.y = h_red, Red.y < Red.N, Red.H, not Blue.H
> h_used = h_red + h_blue + 1
> h_red + h_blue < t_valid
> B[(h_red + h_blue) mod N_B].c = Red
> Red.O[0, Red.y) + [] + (Red.O[Red.y] + 
  B[(h_red + h_blue + 1) redupto t_valid)) + Red.S_IB + Red.S_I = K_Red
Red.H := false
> S, B, I, O, T, Red.y = h_red, Red.y < Red.N, not Red.H, not Blue.H
> h_used = h_red + h_blue + 1
> h_red + h_blue < t_valid
> B[(h_red + h_blue) mod N_B].c = Red
> Red.O[0, Red.y) + [] + (Red.O[Red.y] + 
  B[(h_red + h_blue + 1) redupto t_valid)) + Red.S_IB + Red.S_I = K_Red
h_red := h_red + 1
> S, B, I, O, T, Red.y + 1 = h_red, Red.y < Red.N, not Red.H
> Red.O[0, Red.y) + Red.O[Red.y] + B[(h_red + h_blue) redupto t_valid) + 
  Red.S_IB + Red.S_I = K_Red
Red.y := Red.y + 1
> S, B, I, O, T, Red.y = h_red, Red.y <= Red.N, not Red.H
> Red.O[0, Red.y) + [] + B[(h_red + h_blue) redupto t_valid) + 
  Red.S_IB + Red.S_I = K_Red
\end{verbatim}

\subsection*{Noninterference Proof}
% states after Red.T=false:
% \begin{itemize}
%     \item Blue.CAS doesn't affect:
% if t'=tused: tvalid != tused, precondition impossible
% if t'!=tused: not Blue.T, postcondition compatible
% So OK!
% \end{itemize}

% Blue.CAS doesn't affect states after Red tvalid+=1: simply compatible!

% push() and pop() don't interfere: only makes the predicates stronger

To prove noninterference between colours and between operations, we find that all assertions are a conjunction of predicates of 5 forms:
\begin{itemize}
    \item The invariants S, B, I, O, T, which are proved to hold everywhere as long as other conjuncts hold.
    \item \texttt{\{MyColor\}.\{x|y\} \{<|<=\} \{MyColor\}.N}, \texttt{[not] \{MyColor\}.\{T|H\}}, which only involve local variables and constants, so interference is impossible.
    \item \texttt{\{MyColor\}.y = h\_\{myColor\}}, \texttt{\{MyColor\}.y + 1 = h\_\{myColor\}}, which involve shared variables $h_\text{myColor}$, but they are only ever updated by the concerned process itself, so it cannot be violated by interference.
    \item \texttt{\{Red|Blue\}.\{H|T\} -> not \{Blue|Red\}.\{H|T\}}, which are also parts of the invariant, so already covered.
\end{itemize}

This leaves only the following predicates that might suffer interference:

\begin{enumerate}

\item \texttt{\{MyColor\}.T -> (t\_valid = x\_red + x\_blue)} only appears in \texttt{\{MyColor\}.push()}, and the only non-local statements that might could modify these variables are in  \texttt{\{OtherColor\}.push()}. 

The first such statement is $\mathbf{x_\text{\{otherColor\}}} := x_\text{\{otherColor\}} + 1$, which has $\neg \text{\{MyColor\}}.T$ as precondition and postcondition, so \texttt{\{MyColor\}.T -> (t\_valid = x\_red + x\_blue)} always holds.

The second such statement is $\mathbf{t_\text{valid}} := t_\text{valid} + 1$, where the pre- and post- conditions are also compatible:
\begin{flalign*}
& \triangleright \neg \text{\{MyColor\}}.T \land \neg \text{\{OtherColor\}}.T  &\\
& \triangleright \text{\{MyColor\}}.T \rightarrow (t_\text{valid} = x_\text{red} + x_\text{blue}) \text{  (precondition holds)}  &\\
%
& \mathbf{t_\text{valid}} := t_\text{valid} + 1 & \\
%
& \triangleright \neg \text{\{OtherColor\}}.T &\\
& \triangleright \text{\{OtherColor\}}.T \rightarrow (t_\text{valid} = x_\text{red} + x_\text{blue})  &\\
& \triangleright \text{\{MyColor\}}.T \rightarrow (t_\text{valid} = x_\text{red} + x_\text{blue}) \text{  (no conflict)}  &
\end{flalign*}

\item Similarly, $$\text{\{MyColor\}}.T \land (t_\text{valid} = x_\text{red} + x_\text{blue})$$ and $$\text{\{MyColor\}}.T \land (t_\text{valid} + 1 = x_\text{red} + x_\text{blue})$$ also only appear in \texttt{\{MyColor\}.push()}, and are only possibly affected by the same two statements as above. But this is actually impossible by mutual exclusion, because they both have $\neg \text{\{MyColor\}}.T$ as precondition, which is a conflict.

\item \texttt{\{MyColor\}.T -> (t\_valid + 1 = t\_used)} only appears in \texttt{\{MyColor\}.push()}, and the only statements that might could modify these variables are in  \texttt{\{OtherColor\}.push()}. 

The first such statement is \texttt{CAS(\{OtherColor\}.T, t\_used, t'+1, t')}, whose noninterference can be proved by annotations:
\begin{flalign*}
& \triangleright \text{\{MyColor\}}.T \rightarrow t_\text{valid} + 1 = t_\text{used}  &\\
& \triangleright t' = t_\text{used} \rightarrow t_\text{valid} = t_\text{used} \text{  (precondition no conflict)}  &\\
%
& \text{CAS}(\{\text{OtherColor}\}.T, t_\text{used}, t' + 1, t') & \\
%
& \triangleright \text{\{OtherColor\}}.T \rightarrow \neg \text{\{MyColor\}}.T &\\
& \triangleright \text{\{OtherColor\}}.T \rightarrow t_\text{valid} + 1 = t_\text{used} &\\
& \triangleright \text{\{MyColor\}}.T \rightarrow t_\text{valid} + 1 = t_\text{used} \text{  (postcondition no conflict)}  &
\end{flalign*}

The second such statement is $\mathbf{t_\text{valid}} := t_\text{valid} + 1$, where the pre- and post- conditions are also compatible:
\begin{flalign*}
& \triangleright \text{\{MyColor\}}.T \rightarrow t_\text{valid} + 1 = t_\text{used}  &\\
& \triangleright t_\text{valid} + 1 = t_\text{used}  &\\
& \triangleright \neg \text{\{MyColor\}}.T \land \neg \text{\{OtherColor\}}.T \text{  (precondition no conflict)}  &\\
%
& \mathbf{t_\text{valid}} := t_\text{valid} + 1 & \\
%
& \triangleright \text{\{MyColor\}}.T \rightarrow t_\text{valid} + 1 = t_\text{used}  &\\
& \triangleright \neg \text{\{OtherColor\}}.T  \text{  (postcondition no conflict)}&
\end{flalign*}

\item Similarly, $\neg \text{\{OtherColor\}}.T \land t_\text{valid} + 1 = t_\text{used}$ also only appears in \texttt{\{MyColor\}.push()}, and are only possibly affected by the same two statements as above. 

For \texttt{CAS(\{OtherColor\}.T, t\_used, t'+1, t')}, the noninterference can be proved by annotations:
\begin{flalign*}
& \triangleright \neg \text{\{OtherColor\}}.T \land t_\text{valid} + 1 = t_\text{used}  &\\
& \triangleright t' = t_\text{used} \rightarrow t_\text{valid} = t_\text{used} \ \  (\text{so, } t' \neq t_\text{used})  &\\
%
& \text{CAS}(\{\text{OtherColor}\}.T, t_\text{used}, t' + 1, t') & \\
%
& \triangleright \neg \text{\{OtherColor\}}.T \land t_\text{valid} + 1 = t_\text{used}  \text{  (postcondition no conflict)}  &
\end{flalign*}

For $\mathbf{t_\text{valid}} := t_\text{valid} + 1$:
\begin{flalign*}
& \triangleright \neg \text{\{OtherColor\}}.T \land t_\text{valid} + 1 = t_\text{used}  &\\
& \triangleright \neg \text{\{MyColor\}}.T \land \neg \text{\{OtherColor\}}.T \text{  (precondition no conflict)}  &\\
%
& \mathbf{t_\text{valid}} := t_\text{valid} + 1 & \\
%
& \triangleright \neg \text{\{OtherColor\}}.T\ \ \text{  (postcondition does not care about }t_\text{used}\text{, so no conflict)} &
\end{flalign*}

\item Following the same reasoning in 3 and 4, 
$$\{\text{MyColor}\}.T \rightarrow (t_\text{valid} - h_\text{red} - h_\text{blue} < N_B)$$ 
and 
$$\neg \text{\{OtherColor\}}.T \land (t_\text{valid} - h_\text{red} - h_\text{blue} < N_B) $$ 
in \texttt{\{MyColor\}.push()} cannot be interfered by \texttt{\{OtherColor\}.push()}. In addition, $h_\text{red}$ and $h_\text{blue}$ can be updated by $h_\text{color} := h_\text{color} + 1$ in \texttt{pop()} of both colors, but it only makes the lefthand side of the inequality smaller, so they still holds.

\item $t' = t_\text{used} \rightarrow t_\text{valid} = t_\text{used}$ and $t' = t_\text{used} \rightarrow t_\text{valid} - h_\text{red} - h_\text{blue} < N_B$ only appears in \texttt{push()}, where $t'$ is a local variable. Wherever interference might happen, it is possible to set $t' \neq t_\text{used}$, so it still holds.

\item $$B[(t_\text{valid} + 1)\ \text{\{myColor\}upto}\ (x_\text{red} + x_\text{blue})] = []$$ and
$$B[t_\text{valid}\ \text{\{myColor\}upto}\ (x_\text{red} + x_\text{blue})] = []$$
only appears in \texttt{\{MyColor\}.push()}, and the only statements that might invalidate it is in \texttt{\{OtherColor\}.push()}.

First, $x_\text{\{otherColor\}} := x_\text{\{otherColor\}} + 1$ makes the range larger by 1, but it does not add or remove items of \texttt{myColor}, so they still holds.

Second, $t_\text{valid} := t_\text{valid} + 1$ only makes the range smaller, so they can only stay empty.

\item Following the same reasoning as 6, in \texttt{pop()}, $h' = h_\text{used} \rightarrow h_\text{used} = h_\text{red} + h_\text{blue}$ and $h' = h_\text{used} \rightarrow h_\text{used} < t_\text{valid} \land B[h_\text{used}\ \text{mod}\ N_B].c = \text{Red}$ also hold where interference is possible.

\item Following the same reasoning as in 3 and 4, 
\begin{flalign*}
& \{\text{MyColor}\}.H \rightarrow & \\
& h_\text{used} = h_\text{red} + h_\text{blue} + 1 \land h_\text{red} + h_\text{blue} < t_\text{valid} \land B[(h_\text{red} + h_\text{blue})\ \text{mod}\ N_B].c = \text{Red} &
\end{flalign*}
and 
\begin{flalign*}
& \neg \text{\{OtherColor\}}.H \land & \\
& h_\text{used} = h_\text{red} + h_\text{blue} + 1 \land h_\text{red} + h_\text{blue} < t_\text{valid} \land B[(h_\text{red} + h_\text{blue})\ \text{mod}\ N_B].c = \text{Red} &
\end{flalign*}
in \texttt{\{MyColor\}.pop()} cannot be interfered by \texttt{\{OtherColor\}.pop()}. In addition, $t_\text{valid}$ can be updated by $t_\text{valid} := t_\text{valid} + 1$ in \texttt{push()} of both colors, but it only makes $t_\text{valid}$ larger, so the inequality still holds.

\end{enumerate}

\section*{Partial Correctness}

From initialization $x_\text{\{color\}},h_\text{\{color\}},h_\text{used},t_\text{used},t_\text{valid},\{\text{Color}\}.x,\{\text{Color}\}.y:=0,0,0,0,0,0,0$ and the extended invariant, we know that $ [\ ] + [\ ] + [\ ] + [\ ] + I[0,N) = K_\text{\{Color\}} $, so $ K_\text{\{Color\}} = I[0,N) $.

The extended invariant plus the termination condition ($Red.y=Red.N$) for \textbf{R} (the receiver) implies that 
$$ \{\text{Color}\}.O[0,N) + \{\text{Color}\}.S_{BO}+\{\text{Color}\}.S_{B}+\{\text{Color}\}.S_{IB}+\{\text{Color}\}.S_I = K_\text{\{Color\}},$$ which implies
$$ N + |\{\text{Color}\}.S_{BO}| + |\{\text{Color}\}.S_{B}| + |\{\text{Color}\}.S_{IB}| + |\{\text{Color}\}.S_I| = |\{\text{Color}\}.I[0,N)| = N, $$ therefore
$$ |\{\text{Color}\}.S_{BO}| = |\{\text{Color}\}.S_{B}| = |\{\text{Color}\}.S_{IB}| = |\{\text{Color}\}.S_I| = 0,$$ thus
$$ \{\text{Color}\}.O[0,N) = \{\text{Color}\}.I[0,N).$$
\end{document}

\end{document}
